\documentclass{beamer}
%\documentclass[handout]{beamer}
\usepackage[hungarian]{babel}
\uselanguage{hungarian}
\languagepath{hungarian}
\deftranslation[to=hungarian]{Theorem}{T\'etel}
\deftranslation[to=hungarian]{Example}{P\'elda}
\deftranslation[to=hungarian]{Definition}{Defin\'ici\'o}
%\usepackage[magyar]{babel}
\usepackage[utf8]{inputenc}
\usepackage[T1]{fontenc}
\usepackage{beamerthemesplit}
\usepackage{pgf,pgffor,pgfplots}
\pgfplotsset{compat=1.15}
\usepackage{subfig}
\usepackage{xcolor}
\usepackage{nccfoots}
\newcommand{\framenote}[1]{%
	\Footnotetext{}{\emph{#1}}% Print footnote text
}
\usepackage{lstlinebgrd}
\AtBeginEnvironment{figure}{\setcounter{subfigure}{0}}
\makeatletter
%%%%%%%%%%%%%%%%%%%%%%%%%%%%%%%%%%%%%%%%%%%%%%%%%%%%%%%%%%%%%%%%%%%%%%%%%%%%%%
%
% \btIfInRange{number}{range list}{TRUE}{FALSE}
%
% Test in int number <number> is element of a (comma separated) list of ranges
% (such as: {1,3-5,7,10-12,14}) and processes <TRUE> or <FALSE> respectively

\newcount\bt@rangea
\newcount\bt@rangeb

\newcommand\btIfInRange[2]{%
    \global\let\bt@inrange\@secondoftwo%
    \edef\bt@rangelist{#2}%
    \foreach \range in \bt@rangelist {%
        \afterassignment\bt@getrangeb%
        \bt@rangea=0\range\relax%
        \pgfmathtruncatemacro\result{ ( #1 >= \bt@rangea) && (#1 <= \bt@rangeb) 
        }%
        \ifnum\result=1\relax%
            \breakforeach%
            \global\let\bt@inrange\@firstoftwo%
        \fi%
    }%
    \bt@inrange%
}
\newcommand\bt@getrangeb{%
    \@ifnextchar\relax%
        {\bt@rangeb=\bt@rangea}%
        {\@getrangeb}%
}
\def\@getrangeb-#1\relax{%
    \ifx\relax#1\relax%
        \bt@rangeb=100000%   \maxdimen is too large for pgfmath
    \else%
        \bt@rangeb=#1\relax%
    \fi%
}

%%%%%%%%%%%%%%%%%%%%%%%%%%%%%%%%%%%%%%%%%%%%%%%%%%%%%%%%%%%%%%%%%%%%%%%%%%%%%%
%
% \btLstHL<overlay spec>{range list}
%
% TODO BUG: \btLstHL commands can not yet be accumulated if more than one 
%overlay spec match.
% 
\newcommand<>{\btLstHL}[1]{%
  \only#2{\btIfInRange{\value{lstnumber}}{#1}{\color{orange!30}\def\lst@linebgrdcmd{\color@block}}{\def\lst@linebgrdcmd####1####2####3{}}}%
}%
\makeatother

\usepackage{hyperref}
\hypersetup{
    colorlinks = true,
    linkcolor = blue,
    urlcolor  = blue,
    citecolor = blue,
    linkbordercolor = {white},
}
\usepackage{alltt}
\usepackage{tikz,tkz-euclide}
\usetikzlibrary{trees}
\usetikzlibrary{shapes,shapes.geometric,shapes.multipart}
\usetheme{Warsaw}
\institute{Szegedi Tudományegyetem}
\pgfdeclareimage[height=0.55cm]{institution-logo}{../szte_logo}
\logo{\pgfuseimage{institution-logo}}

\title{Algoritmusok és adatszerkezetek II.}
\subtitle{Geometriai algoritmusok}
\date{}

\begin{document}

\maketitle

\begin{frame}{Alapfogalmak}
\begin{definition}
	A $P_3=\begin{bmatrix} x_{3} \\	y_{3} \end{bmatrix}$ pontot 
	$P_1=\begin{bmatrix} 
	x_{1} \\	y_{1} \end{bmatrix}$ és 
	$P_2=\begin{bmatrix} x_{2} \\	y_{2} \end{bmatrix}$ pontok \textbf{konvex 
	kombináció}jának nevezzük, amennyiben $x_3=(1-\alpha) x_1 + \alpha x_2$, 
	valamint $y_3=(1-\alpha) y_1 + \alpha y_2$ teljesül valamely $0 \leq \alpha 
	\leq 1$-ra
\end{definition}
\begin{definition}
	$\overline{P_1P_2}$ szakasz a $P_1$ és $P_2$ pontokból konvex 
	kombinációinak halmaza
\end{definition}
\begin{alertblock}{Megjegyzés}
	Ha a pontok sorrendje is számít, irányított szakaszról beszélünk, és 
	$\overrightarrow{P_1P_2}$ módon jelöljük \\
	$\vec{p}$-vel $\overrightarrow{OP}$-t, vagyis az $O$ 
	origóból a $P$-be menő irányított szakaszt (vektort) jelöljük
\end{alertblock}
\end{frame}

\begin{frame}{A keresztszorzat}

\begin{block}{$P_1 \times P_2$ keresztszorzata}
	$\det\Bigg(\begin{bmatrix} x_{1} & x_{2} \\	y_1 & y_{2} 
	\end{bmatrix}\Bigg) = x_1y_2-x_2y_1 = P_1 \times P_2 = -P_2 \times P_1$
\end{block}

\begin{block}{Megjegyzés}
	A keresztszorzat valójában háromdimenziós fogalom: egy 
	$\overrightarrow{p_1}$-re és $\overrightarrow{p_2}$-re 
	merőleges, velük jobbsodrású rendszert alkotó vektor, melynek hossza
	$\lvert x_1 y_2 - x_2 y_1 \rvert$.
\end{block}
\end{frame}

\begin{frame}{Forgásirány}
\begin{block}{Keresztszorzat mint előjeles terület}
	$P_1 \times P_2$ megadja az $O$, $P_1$, $P_2$, $P_1+P_2$ koordinátákkal 
	rendelkező paralelogramma előjeles területét
\end{block}
\begin{columns}
	\begin{column}{.1\linewidth}
		\begin{figure}
			\begin{tikzpicture}
				\draw[<->] (0,0)--(2,0) node[right]{$x$};
				\draw[<->] (0,0)--(0,2) node[above]{$y$};
				\draw node[anchor=east] {O};
				\draw[blue,-stealth](0,0)--(.4,1.2) 
				node[anchor=south]{$P_1$};
				\draw[red,-stealth](0,0)--(1.6,.6) 
					node[anchor=north]{$P_2$};
				\draw[red,dashed](1.6,.6)--(2,1.8);
				\draw[blue,dashed](.4,1.2)--(2,1.8) node[anchor=south]
				{\color{black}$P_1+P_2$};
				%\only<2>{
					\draw[black,-stealth](0.745,0.279)--node[right]{m}(.4,1.2);
					\node at (0.85,0.15) {$P$};
				%}
			\end{tikzpicture}
		\end{figure}
	\end{column} \hfill
	\begin{column}{.6\linewidth}
		\begin{itemize}
			\item $\mathbf{P_1 \times P_2 < 0 \Rightarrow P_1}$\textbf{-ből 
			jobbra fordulva érjük el} $\mathbf{P_2}$\textbf{-t}
			\item $P_1 \times P_2 > 0 \Rightarrow P_1$-ből balra fordulva érjük 
			el $P_2$-t
			\item $P_1 \times P_2 = 0 \Rightarrow P_1$ és $P_2$ kollineáris
		\end{itemize}
	\end{column}
\end{columns}
\end{frame}

\begin{frame}{Merre fordul a következő szakasz?}
\begin{itemize}
	\item $\overline{P_0P_1}$ és $\overline{P_1P_2}$ 
	szakaszokat folyamatosan bejárva merre kell fordulni $P_1$ pontban?
	\item Az előzőekben lényegében az origó viselkedett $P_0$-ként
\end{itemize}

\begin{block}<2>{Ötlet: tegyünk úgy, mintha $P_0$ lenne az origó}
	$(P_1-P_0) 
	\times (P_2-P_0) = \det\Bigg(\begin{bmatrix} x_1 - x_0  & x_2 - x_0 \\	y_1 
	- y_0 & y_2 - 
	y_0 
	\end{bmatrix}\Bigg)$ % = (x_1-x_0)(y_2-y_0)-(x_2-x_0)(y_1-y_0)$
\end{block}

\begin{itemize}
	\item<2> Szemléletesen: $P_1$-ből és $P_2$-ből $P_0$-t kivonva $P_0$ 
	központúvá tesszük a koordinátarendszerünket
\end{itemize}

\end{frame}

\begin{frame}{Szakasz átfogása}

\begin{block}{Átfogó szakasz}
Egy $\overline{P_1P_2}$ szakasz átfog egy egyenest, ha a $P_1$ pont az egyenes
egyik oldalára, $P_2$ pont pedig a másik oldalára esik
\end{block}
\begin{figure}
			\begin{tikzpicture}[scale=.7]
	\tkzInit[xmax=5,ymax=3,xmin=0,ymin=0]
	\tkzAxeXY
	\node (E) at (5.5,2.5) {};
	\node (F) at (-.5,.5) {};	
	\node[outer sep=0pt,circle, fill,inner 
	sep=2pt,label={above:$P_1$}] (C) at (2,3) {};
	\node[outer sep=0pt,circle, fill,inner sep=2pt, 
	label={above:$P_2$}] (D) at (5,1) {};
	
    \draw[dashed,red] (E)--(F);
	\draw[line width=.1em] (C)--(D);
	\end{tikzpicture}
\end{figure}

\begin{block}<2>{Átfedés meglétének eldöntése}
	Egy (kevéssé hatékony) lehetőség, ha az egyenes egyenletét kiszámolva 
	döntünk $P_1$ és $P_2$ relatív helyzetéről \\
	\textbf{Támaszkodjunk helyette a forgásirányokra!}
\end{block}

\end{frame}

\begin{frame}{Egymást metsző szakaszok}
\begin{block}{Szükségesség}
	$\overline{CD}$ úgy metszheti $\overline{AB}$ szakaszt, ha $\overline{CD}$ 
	átfogja az $\overline{AB}$ szakaszra illeszkedő egyenest.
\end{block}

\begin{columns}
	\begin{column}{.5\linewidth}
		\begin{tikzpicture}[scale=.8]
		\tkzInit[xmax=5,ymax=3,xmin=0,ymin=0]
		\tkzAxeXY
		\node[outer sep=0pt,circle, fill,inner 
		sep=2pt,label={above:$A$}] (A) at (1,1) {};
		\node[outer sep=0pt,circle, fill,inner sep=2pt, 
		label={above:$B$}] (B) at (4,2) {};
		\node (E) at (5.5,2.5) {};
		\node (F) at (-.5,.5) {};	
		\node[outer sep=0pt,circle, fill,inner 
		sep=2pt,label={above:$C$}] (C) at (2,3) {};
		\node[outer sep=0pt,circle, fill,inner sep=2pt, 
		label={above:$D$}] (D) at (5,1) {};
		
		\only<2>{\draw[dashed,red] (E)--(F)};
		\draw[line width=.1em] (A)--(B);
		\draw[line width=.1em] (C)--(D);
		\end{tikzpicture}
	\end{column}
	\begin{column}<2>{.5\linewidth}
		\begin{tikzpicture}[scale=.8]
		\tkzInit[xmax=5,ymax=3,xmin=0,ymin=0]
		\tkzAxeXY
		\node[outer sep=0pt,circle, fill,inner 
		sep=2pt,label={above:$A$}] (A) at (1,1) {};
		\node[outer sep=0pt,circle, fill,inner sep=2pt, 
		label={below:$B$}] (B) at (3.25,1.75) {};
		\node (E) at (5.5,2.5) {};
		\node (F) at (-.5,.5) {};	
		\node[outer sep=0pt,circle, fill,inner 
		sep=2pt,label={above:$C$}] (C) at (2,3) {};
		\node[outer sep=0pt,circle, fill,inner sep=2pt, 
		label={above:$D$}] (D) at (5,1) {};
		
		\only<2>{\draw[dashed,red] (E)--(F)};
		\draw[line width=.1em] (A)--(B);
		\draw[line width=.1em] (C)--(D);
		\end{tikzpicture}
	\end{column}
\end{columns}
\end{frame}

\begin{frame}[fragile]{Metszés vizsgálata}
\begin{columns}
	\begin{column}{.6\linewidth}
		\begin{alltt}
			{\scshape Forgásirány}(X, Y, Z) \{
			   return (Y-X)\(\times\)(Z-X)
			\} \vfill
			{\scshape MetszőSzakaszok}(A, B, C, D) \{
			   d1 = {\scshape Forgásirány}(A, B, C)
			   d2 = {\scshape Forgásirány}(A, B, D)
			   d3 = {\scshape Forgásirány}(C, D, A)
			   d4 = {\scshape Forgásirány}(C, D, B)
			   return d1 * d2 < 0 és d3*d4 < 0
			\}
		\end{alltt}
	\end{column}
	\begin{column}<2>{.4\linewidth}
		Ezzel csak ``valódi'' metszéseket találunk meg, a szakaszra 
		illeszkedő végpontú szakaszt nem kezeltük így
	\end{column}
\end{columns}
\end{frame}

\end{document}