\documentclass{beamer}
%\documentclass[handout]{beamer}
\usepackage[hungarian]{babel}
\uselanguage{hungarian}
\languagepath{hungarian}
\deftranslation[to=hungarian]{Theorem}{T\'etel}
\deftranslation[to=hungarian]{Example}{P\'elda}
%\usepackage[magyar]{babel}
\usepackage[utf8]{inputenc}
\usepackage[T1]{fontenc}
\usepackage{beamerthemesplit}
\usepackage{pgf,pgffor,pgfplots}
\pgfplotsset{compat=1.15}
\usepackage{subfig}
\usepackage{listings}
\usepackage{lstlinebgrd}
\usepackage{etoolbox}% http://ctan.org/pkg/etoolbox
\AtBeginEnvironment{figure}{\setcounter{subfigure}{0}}
\makeatletter
%%%%%%%%%%%%%%%%%%%%%%%%%%%%%%%%%%%%%%%%%%%%%%%%%%%%%%%%%%%%%%%%%%%%%%%%%%%%%%
%
% \btIfInRange{number}{range list}{TRUE}{FALSE}
%
% Test in int number <number> is element of a (comma separated) list of ranges
% (such as: {1,3-5,7,10-12,14}) and processes <TRUE> or <FALSE> respectively

\newcount\bt@rangea
\newcount\bt@rangeb

\newcommand\btIfInRange[2]{%
    \global\let\bt@inrange\@secondoftwo%
    \edef\bt@rangelist{#2}%
    \foreach \range in \bt@rangelist {%
        \afterassignment\bt@getrangeb%
        \bt@rangea=0\range\relax%
        \pgfmathtruncatemacro\result{ ( #1 >= \bt@rangea) && (#1 <= \bt@rangeb) }%
        \ifnum\result=1\relax%
            \breakforeach%
            \global\let\bt@inrange\@firstoftwo%
        \fi%
    }%
    \bt@inrange%
}
\newcommand\bt@getrangeb{%
    \@ifnextchar\relax%
        {\bt@rangeb=\bt@rangea}%
        {\@getrangeb}%
}
\def\@getrangeb-#1\relax{%
    \ifx\relax#1\relax%
        \bt@rangeb=100000%   \maxdimen is too large for pgfmath
    \else%
        \bt@rangeb=#1\relax%
    \fi%
}

%%%%%%%%%%%%%%%%%%%%%%%%%%%%%%%%%%%%%%%%%%%%%%%%%%%%%%%%%%%%%%%%%%%%%%%%%%%%%%
%
% \btLstHL<overlay spec>{range list}
%
% TODO BUG: \btLstHL commands can not yet be accumulated if more than one overlay spec match.
% 
\newcommand<>{\btLstHL}[1]{%
  \only#2{\btIfInRange{\value{lstnumber}}{#1}{\color{orange!30}\def\lst@linebgrdcmd{\color@block}}{\def\lst@linebgrdcmd####1####2####3{}}}%
}%
\makeatother



\usepackage{hyperref}
\hypersetup{
    colorlinks = true,
    linkcolor = blue,
    urlcolor  = blue,
    citecolor = blue,
    linkbordercolor = {white},
}
\usepackage{alltt}
\usepackage{tikz}
\usetikzlibrary{trees}
\usetikzlibrary{shapes,shapes.geometric,shapes.multipart}
\usetikzlibrary{calc,chains,arrows,positioning}
\tikzset{
  ownnode/.style={fill,circle,inner sep=2pt,outer sep=0pt},
  box/.style={draw, fill=pink!10, minimum width=5em, text centered, minimum height=2.5em},
  treenode/.style = {circle, draw, align=center, inner sep=3pt, text centered, font=\sffamily, text width=1em},
  int/.style = {rectangle split,rectangle split parts=2,draw,text centered, text width = 1.6cm, text height = 0.3cm},
  subtree/.style={isosceles triangle, draw=black, align=center, minimum height=0.5cm, minimum width=1cm, shape border rotate=90, anchor=north},
  data/.style={
      minimum width=2em,
      minimum height=2em,
      draw, rectangle split,
      rectangle split parts=2, text centered,
  }
}

\usepackage{animate}
\usepackage{clrscode3e}
\usetheme{Warsaw}
\institute{Szegedi Tudományegyetem}
\pgfdeclareimage[height=0.55cm]{institution-logo}{../szte_logo}
\logo{\pgfuseimage{institution-logo}}

\title{Algoritmusok és adatszerkezetek II.}
\subtitle{Kiegyensúlyozott keresőfák}
\date{}

\begin{document}

\maketitle

\begin{frame}{Mit értünk kiegyensúlyozott keresőfa alatt?}
	\begin{itemize}
		\item Eddigiekben láttuk, hogy a(z augmentált) keresőfák esetében érintett műveletek mindegyike $O(h)$ volt (ahol $h$ a fa magassága)
		\item Ezen felül tudjuk azt is, hogy $h=O(n)$ 
		\item Hogy keresnénk meg egy bináris keresőfa $r$ rangú elemét?
		\item Hogyan határoznánk meg egy bináris keresőfa $x$ csúcsának rangját?
	\end{itemize}
	\pause
	\begin{beamerboxesrounded}{Általános ötlet}
		A fapontokban tárolt extra adattagok hozzásegíthetnek minket új műveletek hatékony ($O(h)$ idejű) végrehajtásához
	\end{beamerboxesrounded}
\end{frame}

\begin{frame}
  \animategraphics[loop,controls,width=\linewidth]{12}{avl/avl-}{0}{16}
\end{frame}

\begin{frame}{Új műveletek -- adott rangú kulcs meghatározása}
	\begin{itemize}
		\item Cél: $r \leq n$ ranggal rendelkező elem megtalálása a fában
		\item Naiv (de működő) elgondolás: elkezdem bejárni a fát $<$ szerinti sorrendben, és megállok az $r$-ediknek érintett csúcsnál
		\item<2> $O(h)$ helyett $\Theta(r)$ idejű algoritmus
		\begin{itemize}
			\item Kiegyensúlyozott fa és kellően nagy $r$ estében pedig $r \gg h$
		\end{itemize}
	\end{itemize}
\end{frame}

\begin{frame}{Új műveletek -- adott kulcs rangjának meghatározása}
	\begin{itemize}
		\item Naiv elgondolás: elkezdem bejárni a fát $<$ szerinti sorrendben, és megállok, ha $x$ kulcsot érintem
		\item A válasz az $x$ megtalálásáig érintett kulcsok száma
		\item $O(h)$ helyett $O(n)$ idejű algoritmus
	\end{itemize}
	\begin{alertblock}<2>{A megoldás}
Minden csúcs tároljon el kiegészítő információt magáról!
	\end{alertblock}
\end{frame}

\section{Rendezettminta-fa}
\begin{frame}[fragile]{Rendezett-minta fa implementációja}
	\begin{columns}
		\begin{column}{.7\textwidth}
\begin{lstlisting}[
  linebackgroundcolor={
    \btLstHL<1->{3}
  }]
class Node {
    Object kulcs;
    int kiegeszito;
    Node* apa;
    Node* bal;
    Node* jobb;
}
\end{lstlisting}

			\begin{block}<2>{Megjegyzés}
				A kiegészítő információ legyen az adott gyökerű részfa mérete (benne található kulcsok száma). \\
				Üres fa kiegészítő információja 0.
			\end{block}
		\end{column}
		
		\begin{column}{.4\textwidth}
		\begin{figure}
		\centering
\begin{tikzpicture}[node distance=0cm,outer sep = 0pt]
    \node (A) [box,minimum width=6em] {kulcs};
    \node (H) [box,anchor=north,minimum width=6em,fill=orange!30] at (A.south) {kiegeszito};
    \node (D) [box,anchor=south,minimum width=6em] at (A.north) {apa};
    \node (B) [box,anchor=north west,minimum width=3em] at (H.south west) {bal};
    \node (C) [box,anchor=north east,minimum width=3em] at (H.south east) {jobb};
    \coordinate (E) at (0,2);
    \coordinate (E) at (0,2);
    \node[below left = 1cm of B] (F) {};
    \node[below right = 1cm of C] (G) {};
%    \path (D) edge [->] node[pos=0.5,anchor=-135,inner sep=1pt] {N} (E);
    \path (D) edge [->] (E);
    \path (B) edge [->] (F);
    \path (C) edge [->] (G);
\end{tikzpicture}
		\end{figure}
		\end{column}
	\end{columns}
\end{frame}

\begin{frame}[fragile]{Adott rangú kulcs keresése}
	\begin{columns}
		\begin{column}{.6\linewidth}
\begin{alltt}
{\scshape RangKeres}(x, i) \{
   r = x.bal.kiegeszito + 1

   if (i < r) \{
      {\scshape RangKeres}(x.bal, i)
   \} elseif (i > r) \{
      {\scshape RangKeres}(x.jobb, i - r)
   \} else \{
      return x
   \}
 \}
\end{alltt}
		\end{column}
		\begin{column}{.35\linewidth}
			$h$ magas fa esetén $O(h)$
		\end{column}
	\end{columns}
\end{frame}

\begin{frame}[fragile]{Adott kulcs rangjának meghatározása}
\begin{alltt}
{\scshape RangMeghatároz}(x) \{
   r = 0
   y = x
   while (y != nil) \{
       if (x.kulcs >= y.kulcs) \{
           r = r + y.bal.kiegeszito + 1
       \}
       y = y.apa
   \}
   return r
 \}
\end{alltt}
\begin{block}{Szövegesen}
$x$-től a gyökérig lépkedve azon $y$ gyökerű részfák gyökérelemeinek rangjait összegezzük, melyekre $x.kulcs >= y.kulcs$
\end{block}
\end{frame}

\begin{frame}[fragile]{A kiegészítő információk fenntartása}
	\begin{itemize}
		\item A csúcsokban tárolt kiegészítő információknak mindig naprakészeknek kell legyenek
	\end{itemize}
	\begin{block}{Példa: {\scshape Töröl(40)}}
		\begin{figure}
		\centering
			\begin{tikzpicture}[level/.style={sibling distance = 3cm},level distance = 1cm]
				\only<1>{
					\node[data]{40 \nodepart{second}6}
						child{ node[data]{20 \nodepart{two}3}
							child{ node[data]{13 \nodepart{two}1}}
							child{ node[data]{33 \nodepart{two}1}}
						}
						child{ node[data]{52 \nodepart{two}2}
							child[missing]
							child{ node[data]{66 \nodepart{two}1}}
						};
				}
				\only<2->{
				      \node[data]{33 \nodepart{two}{\textcolor{red}{5}}}
				          child{ node[data]{20 \nodepart{two}{\textcolor{red}{2}}}
				      	    child{ node[data]{13 \nodepart{two}1}}
				            child[missing]
				          }
				          child{ node[data]{52 \nodepart{two}2}
				              child[missing]
				              child{ node[data]{66 \nodepart{two}1}}
				          };
				}
			\end{tikzpicture}
		\end{figure}
		\begin{alertblock}<3>{Észrevétel}
			Épp ezért nem a csúcsok rangjára tekintünk közvetlenül kiegészítő információként (hiszen azt költséges lehet aktualizálni)
		\end{alertblock}		
	\end{block}
\end{frame}

\begin{frame}[fragile]{Új művelet -- Forgatás}
  \begin{itemize}
    \item A keresőfákat forgatásokkal fogjuk tudni kiegyensúlyozni
    \begin{figure}
    \subfloat[$x$ körüli balra forgatás előtt]{
      \begin{tikzpicture}[level/.style={sibling distance = 2cm},level distance=1cm]
          \node[data]{x \nodepart{two}{20}}
              child{ node[subtree]{$\alpha$\\ 6}
              }
              child{ node[data]{y \nodepart{two}{13}}
                  child{ node[subtree]{$\beta$\\ 4}}
                  child{ node[subtree]{$\gamma$\\ 8}}
              };
        \end{tikzpicture}
    } \hfill
    \only<2->{
    \subfloat[$x$ körüli balra forgatás után]{
    	\begin{tikzpicture}[level/.style={sibling distance = 2cm},level distance=1cm]
    		\node[data]{y \nodepart{two} \only<-2>{\phantom{20}}\only<3>{20}}
    		child{ node[data]{x \nodepart{two} \only<-2>{\phantom{11}}\only<3>{11}}
    		                  child{ node[subtree]{$\alpha$\\ \only<3>{6}}}
    		                  child{ node[subtree]{$\beta$\\ \only<3>{4}}}
    		 }
            child{ node[subtree]{$\gamma$\\\only<3>{8}}}
            ;
            \end{tikzpicture}
        }
    }
    \end{figure}
  \end{itemize}
\end{frame}

\begin{frame}{A kiegészítő információ fenntartása}
	\begin{itemize}
		\item Beszúrás esetén: a beszúrás helyétől a gyökérig menően a kiegészítő információk inkrementálása ($O(h)$)
		\item Törlés esetén: a kieső csúcstól a gyökéig menően a kiegészítő információk dekrementálása ($O(h)$)
		\item $x$ körüli forgatás esetén ($O(1)$)
		\begin{itemize}
			\item $y.kiegeszito = x.kiegeszito$
			\item $x.kiegeszito = x.bal.kiegeszito + y.bal.kiegeszito + 1$, azaz $\alpha$-beli és $\beta$-beli csúcsok száma +1
		\end{itemize}
	\end{itemize}
\end{frame}

\section{Intervallumfák}
\begin{frame}{Intervallumfák}
	\begin{block}{Az intervallumfa sajátosságai}
		\begin{itemize}
			\item A csúcsok $[a;f]$ intervallumokat tárolnak
			\begin{itemize}
				\item $a$ és $f$ az intervallum alsó,-és felső végpontjait jelöli
				\item $a \leq f$ feltehető
			\end{itemize}
			\item $<$ rendezést az intervallumok kezdőpontja szerint értelmezzük
			\begin{itemize}
				\item $[1;4] < [5;6]$
				\item $[1;4] < [3;5]$
				\item $[1;4] < [2;3]$
			\end{itemize}
		\end{itemize}
	\end{block}
  
	\begin{block}<2->{Speciális művelet: átfedő intervallum keresése}
		El akarjuk tudni dönteni, hogy a fa tartalmaz-e valamely $I=[i_a;i_f]$ intervallummal átfedő intervallumot.\\
		$I$ és $J$ intervallumok átfednek $\Leftrightarrow i_a \leq j_f$ és $i_f \geq j_a$
	\end{block}
\end{frame}


\begin{frame}{Intervallum trichotómia}

	\begin{itemize}
		\item Az alábbi állítások közül pontosan egy teljesül bármely $(I,J)$ intervallumpárra
		\begin{enumerate}
			\item[a] $I \cap J \neq \emptyset$, azaz $I$ és $J$ intervallumok átfedik egymást
			\item[b] $i_a > j_f$
			\item[c] $i_f < j_a$
		\end{enumerate}
	\end{itemize}
	
\only<1>{
	\begin{block}{Illusztráció -- 1. eset}
		\begin{figure}
			\centering
			\subfloat[Átfedő I és J]{
			\begin{tikzpicture}
				\draw(0,0pt)--(2,0pt);
				\foreach \x in {0,...,2}
					\draw(\x,0pt)--(\x,-3pt);
				\draw (0,6pt) -- (1,6pt)
					node[ownnode,pos=0,label=above:{$i_a$}]{}
					node[ownnode,pos=1,label=above:{$i_f$}]{};
				\draw (0.5,30pt) -- (2,30pt)
					node[ownnode,pos=0,label=above:{$j_a$}]{}
					node[ownnode,pos=1,label=above:{$j_f$}]{};
			\end{tikzpicture}

			\begin{tikzpicture}
				\draw(0,0pt)--(2,0pt);
				\foreach \x in {0,...,2}
					\draw(\x,0pt)--(\x,-3pt);
				\draw (0.5,6pt) -- (2,6pt)
					node[ownnode,pos=0,label=above:{$i_a$}]{}
					node[ownnode,pos=1,label=above:{$i_f$}]{};
				\draw (0,30pt) -- (1.6,30pt)
					node[ownnode,pos=0,label=above:{$j_a$}]{}
					node[ownnode,pos=1,label=above:{$j_f$}]{};
			\end{tikzpicture}

			\begin{tikzpicture}
				\draw(0,0pt)--(2,0pt);
				\foreach \x in {0,...,2}
					\draw(\x,0pt)--(\x,-3pt);
				\draw (0.5,6pt) -- (1.2,6pt)
					node[ownnode,pos=0,label=above:{$i_a$}]{}
					node[ownnode,pos=1,label=above:{$i_f$}]{};
				\draw (0,30pt) -- (2,30pt)
					node[ownnode,pos=0,label=above:{$j_a$}]{}
					node[ownnode,pos=1,label=above:{$j_f$}]{};
			\end{tikzpicture}

			\begin{tikzpicture}
				\draw(0,0pt)--(2,0pt);
				\foreach \x in {0,...,2}
					\draw(\x,0pt)--(\x,-3pt);
				\draw (0,6pt) -- (2,6pt)
					node[ownnode,pos=0,label=above:{$i_a$}]{}
					node[ownnode,pos=1,label=above:{$i_f$}]{};
				\draw (0.6,30pt) -- (1.1,30pt)
					node[ownnode,pos=0,label=above:{$j_a$}]{}
					node[ownnode,pos=1,label=above:{$j_f$}]{};
			\end{tikzpicture}
			}
		\end{figure}
	\end{block}
}
\only<2>{
	\begin{block}{Illusztráció -- 2. és 3. eset}
		\begin{figure}
			\setcounter{subfigure}{1}
			\centering
			\subfloat[$i_a > j_f$]{
			\begin{tikzpicture}
				\draw(0,0pt)--(3,0pt);
				\foreach \x in {0,...,3}
					\draw(\x,0pt)--(\x,-3pt);
				\draw (1.6,6pt) -- (2.6,6pt)
					node[ownnode,pos=0,fill=red,label=above:{$i_a$}]{}
					node[ownnode,pos=1,label=above:{$i_f$}]{};
				\draw (0,30pt) -- (0.8,30pt)
					node[ownnode,pos=0,label=above:{$j_a$}]{}
					node[ownnode,pos=1,fill=red,label=above:{$j_f$}]{};
			\end{tikzpicture}
			} \hfil
			\subfloat[$i_f < j_a$]{
			\begin{tikzpicture}
				\draw(0,0pt)--(3,0pt);
				\foreach \x in {0,...,3}
					\draw(\x,0pt)--(\x,-3pt);
				\draw (0,6pt) -- (0.8,6pt)
					node[ownnode,pos=0,label=above:{$i_a$}]{}
					node[ownnode,pos=1,fill=red,,label=above:{$i_f$}]{};
				\draw (1.8,30pt) -- (2.6,30pt)
					node[ownnode,pos=0,fill=red,label=above:{$j_a$}]{}
					node[ownnode,pos=1,label=above:{$j_f$}]{};
			\end{tikzpicture}
			}
		\end{figure}
	\end{block}
}
\end{frame}

\begin{frame}[fragile]{Intervallumfa implementációja}
	\begin{columns}
		\begin{column}{.7\textwidth}
\begin{lstlisting}[
  linebackgroundcolor={
    \btLstHL<1->{2-3}
  }]
class Node {
    int also;
    int felso;
    int kiegeszito;
    Node* apa;
    Node* bal;
    Node* jobb;
}
\end{lstlisting}
			\begin{block}<2>{Ötlet}
				A kiegészítő információ legyen az adott gyökerű részfában található maximális felső végpont
			\end{block}
		\end{column}
		
		\begin{column}{.4\textwidth}
			\begin{figure}
				\centering
				\begin{tikzpicture}[node distance=0cm,outer sep = 0pt]
				    \node (A) [box,minimum width=6em,fill=orange!30] {also};
				    \node (I) [box,anchor=north,minimum width=6em,fill=orange!30] at (A.south) {felso};
				    \node (H) [box,anchor=north,minimum width=6em] at (I.south) {kiegeszito};
			    	\node (D) [box,anchor=south,minimum width=6em] at (A.north) {apa};
				    \node (B) [box,anchor=north west,minimum width=3em] at (H.south west) {bal};
				    \node (C) [box,anchor=north east,minimum width=3em] at (H.south east) {jobb};
				    \coordinate (E) at (0,2);
				    \coordinate (E) at (0,2);
			    	\node[below left = 1cm of B] (F) {};
				    \node[below right = 1cm of C] (G) {};
				%    \path (D) edge [->] node[pos=0.5,anchor=-135,inner sep=1pt] {N} (E);
				    \path (D) edge [->] (E);
				    \path (B) edge [->] (F);
			    	\path (C) edge [->] (G);
				\end{tikzpicture}
			\end{figure}
		\end{column}
	\end{columns}
\end{frame}

\begin{frame}[fragile]{Intervallumfa -- példa}
	\begin{figure}
		\centering
	    \begin{tikzpicture}[level/.style={sibling distance = 7cm/#1},level distance = 1.2cm]
	    \node[data]{[6;7]\nodepart{two}11}
	        child{
	            node[data,color=purple]{[4;5]\nodepart{two}6}
	            child{node[data,color=orange] {[0;1]\nodepart{two}2}}
	            child{node[data,color=olive] {[4;6]\nodepart{two}6}}
	            }
	        child{
	            node[data,color=red]{[8;11]\nodepart{two}11}
	            child{node[data,color=blue]{[6;8]\nodepart{two}8}}
	            child{node[data,color=cyan]{[10;10]\nodepart{two}10}}
	        };
	    \end{tikzpicture}

   		\vskip 0.8cm
    
		\begin{tikzpicture}[thick]
			\draw(0,0pt)--(11,0pt);
			\foreach \x in {0,1,...,10}
				\draw(\x,0pt)--(\x,-3pt) node[below] {$\x$};
			\draw[orange] (0,6pt) -- (1,6pt)
				node[ownnode,pos=0]{}
				node[ownnode,pos=1]{};
			\draw[purple] (4,12pt) -- (5,12pt)
				node[ownnode,pos=0]{}
				node[ownnode,pos=1]{};
			\draw[olive] (4,18pt) -- (6,18pt)
				node[ownnode,pos=0]{}
				node[ownnode,pos=1]{};
			\draw (6,24pt) -- (7,24pt)
				node[ownnode,pos=0]{}
				node[ownnode,pos=1]{};
			\draw[blue] (6,30pt) -- (8,30pt)
				node[ownnode,pos=0]{}
				node[ownnode,pos=1]{};
			\draw[red] (8,36pt) -- (11,36pt)
				node[ownnode,pos=0]{}
				node[ownnode,pos=1]{};
			\draw[cyan] (10,42pt) -- (10,42pt)
				node[ownnode,pos=0]{}
				node[ownnode,pos=1]{};
		\end{tikzpicture}
	\end{figure}
\end{frame}

\begin{frame}[fragile]{Átfedő intervallum keresése $x$ gyökerű részfában}
\begin{alltt}
{\scshape ÁtfedőKeres}(x, i) \{
   while (x != nil) \{
     if (i.also <= x.felso és i.felso >= x.also) \{
        return x    // átfedést találtunk
     \}
     if (x.bal != nil és x.bal.kiegeszito >= i.also) \{
       x = x.bal    // folytassuk balra a keresést
     \} else \{
       x = x.jobb   // folytassuk jobbra a keresést
     \}
   \}
   return nil       // nem találtunk átfedő intervallumot
 \}
\end{alltt}
\end{frame}

\begin{frame}{Átfedő intervallum keresésének helyessége}
	\begin{theorem}
	Az {\scshape ÁtfedőKeres}(x,i) minden végrehajtása csak abban az esetben tér vissza \texttt{nil}-lel, ha az $x$ gyökerű intervallumfában nem található $i$-vel átfedő intervallum.
	\end{theorem}
\end{frame}

\begin{frame}{Átfedő intervallum keresésének helyessége -- jobbra lépés}
	\begin{itemize}
		\item Akkor lépünk jobbra, ha
		\begin{itemize}
			\item Nincs bal részfa, vagy 
			\item Van bal részfa, de még a legnagyobb felső végpont is elmarad $i$ intervallum alsó végpontjától (trichotómia b) esete)
		\end{itemize}
		\pause $\Rightarrow$ jobbra lépéskor a bal részfában biztosan nincs átfedés
		\pause
		\begin{alertblock}{Összegezve}
			Vagy fogunk jobbra lépve átfedő intervallumot találni, vagy balra lépve se találtunk volna
		\end{alertblock}
	\end{itemize}
\end{frame}

\begin{frame}{Átfedő intervallum keresésének helyessége -- balra lépés}
	\begin{itemize}
		\item Előfordulhat-e, hogy $x$-nél balra menve nem találunk $i$-vel átfedő intervallumot, jobbra menve azonban találhatnánk?
		\begin{itemize}
		    \pause
			\item Indirekt bizonyítás: tegyük fel, hogy előfordulhat ilyen
			\begin{itemize}
				\item Balra mentünk, tehát $\exists y \in x.bal$\footnote{$x.bal$ az $x$ csúcs bal fiát a gyökeréül tudó részfát jelöli}, melyre $y.felso > i.also$
			    \pause
				\item Feltevésünk szerint $x.bal$-ban nincs $i$-vel átfedő intervallum $\rightarrow$ trichotómia c) esete \pause{$\rightarrow y.also > i.felso$}
				\item A keresőfa-tulajdonságból adódóan pedig $\forall y' \in x.jobb$ esetében $y'.also > y.also$ teljesül
				\pause
				\item Tehát $\forall y' \in x.jobb$ az $i$ intervallumra nézve c)-típusú (vele át nem fedő) lehet csupán
			\end{itemize}				
		\end{itemize}
		\begin{alertblock}{Összegezve}
			Ha balra lépve nem sikerül átfedő intervallumot találjuk, akkor jobbra lépve se találhattunk volna
		\end{alertblock}
	\end{itemize}
\end{frame}

\begin{frame}{Összegzés}
	\begin{itemize}
		\item Addicionális adattagok bevezetésével újfajta kérdések hatékony megválaszolása válhat lehetővé
		\item Adatszerkezetek kibővítésének fő lépései
		\begin{enumerate}
			\item Alap-adatszerkezet megválasztása (nem csak bináris keresőfa)
			\item Alap-adatszerkezet kiegészítő információjának kijelölése
			\item Kiegészítő információjának \textbf{hatékony} fenntarthatóságának igazolása
			\item Új hatékony műveletek kifejlesztése
		\end{enumerate}
	\end{itemize}
\end{frame}

\end{document}