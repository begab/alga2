\documentclass{beamer}
%\documentclass[handout]{beamer}
\usepackage[hungarian]{babel}
\uselanguage{hungarian}
\languagepath{hungarian}
\deftranslation[to=hungarian]{Example}{P\'elda}
%\usepackage[magyar]{babel}
\usepackage[utf8]{inputenc}
\usepackage[T1]{fontenc}
\usepackage{beamerthemesplit}
\usepackage{pgf,pgffor}
\usepackage{listings}
\usepackage{lstlinebgrd}
\makeatletter
%%%%%%%%%%%%%%%%%%%%%%%%%%%%%%%%%%%%%%%%%%%%%%%%%%%%%%%%%%%%%%%%%%%%%%%%%%%%%%
%
% \btIfInRange{number}{range list}{TRUE}{FALSE}
%
% Test in int number <number> is element of a (comma separated) list of ranges
% (such as: {1,3-5,7,10-12,14}) and processes <TRUE> or <FALSE> respectively

\newcount\bt@rangea
\newcount\bt@rangeb

\newcommand\btIfInRange[2]{%
    \global\let\bt@inrange\@secondoftwo%
    \edef\bt@rangelist{#2}%
    \foreach \range in \bt@rangelist {%
        \afterassignment\bt@getrangeb%
        \bt@rangea=0\range\relax%
        \pgfmathtruncatemacro\result{ ( #1 >= \bt@rangea) && (#1 <= \bt@rangeb) }%
        \ifnum\result=1\relax%
            \breakforeach%
            \global\let\bt@inrange\@firstoftwo%
        \fi%
    }%
    \bt@inrange%
}
\newcommand\bt@getrangeb{%
    \@ifnextchar\relax%
        {\bt@rangeb=\bt@rangea}%
        {\@getrangeb}%
}
\def\@getrangeb-#1\relax{%
    \ifx\relax#1\relax%
        \bt@rangeb=100000%   \maxdimen is too large for pgfmath
    \else%
        \bt@rangeb=#1\relax%
    \fi%
}

%%%%%%%%%%%%%%%%%%%%%%%%%%%%%%%%%%%%%%%%%%%%%%%%%%%%%%%%%%%%%%%%%%%%%%%%%%%%%%
%
% \btLstHL<overlay spec>{range list}
%
% TODO BUG: \btLstHL commands can not yet be accumulated if more than one overlay spec match.
% 
\newcommand<>{\btLstHL}[1]{%
  \only#2{\btIfInRange{\value{lstnumber}}{#1}{\color{orange!30}\def\lst@linebgrdcmd{\color@block}}{\def\lst@linebgrdcmd####1####2####3{}}}%
}%
\makeatother



\usepackage{hyperref}
\hypersetup{
    colorlinks = true,
    linkcolor = blue,
    urlcolor  = blue,
    citecolor = blue,
    linkbordercolor = {white},
}
\usepackage{alltt}
\usepackage{tikz}
\usetikzlibrary{shapes,shapes.geometric,shapes.multipart}
\usetikzlibrary{calc,chains,arrows,positioning}
\tikzset{
  box/.style={draw, fill=pink!10, minimum width=5em, text centered, minimum height=2.5em},
  treenode/.style = {circle, draw, align=center, inner sep=3pt, text centered, font=\sffamily, text width=1em},
  arn_n/.style = {treenode, circle, white, font=\sffamily\bfseries, draw=black, fill=black, text width=1.5em},
  arn_r/.style = {treenode, circle, red, draw=red, text width=1.5em, very thick},
  rendmint/.style = {rectangle split,rectangle split parts=2,draw,text centered},
  int/.style = {rectangle split,rectangle split parts=2,draw,text centered, text width = 1.6cm, text height = 0.3cm},
  subtree/.style={isosceles triangle, draw=black, align=center, minimum height=0.5cm, minimum width=1cm, shape border rotate=90, anchor=north}
}

\usepackage{clrscode3e}
\usetheme{Warsaw}
\institute{Szegedi Tudományegyetem}
\pgfdeclareimage[height=0.55cm]{institution-logo}{../szte_logo}
\logo{\pgfuseimage{institution-logo}}

\title{Algoritmusok és adatszerkezetek II.}
\subtitle{Augmentált keresőfák}
\date{}

\begin{document}

\maketitle

\begin{frame}{Bináris keresőfák műveletei}
	\begin{itemize}
		\item $h$ magas és $n$ csúcsból álló bináris keresőfa esetén
		a {\scshape Keres}, {\scshape Beszúr}, {\scshape Töröl} műveletek $O(h)$ időben végrehajthatók
		\item $x$ csúcs rangja: a fában tárolt kulcsok között $x$ hanyadik helyen áll $<$ szerint
		\item Hogy keresnénk meg egy bináris keresőfa $r$ rangú elemét?
		\item Hogyan határoznánk meg egy bináris keresőfa $x$ csúcsának rangját?
	\end{itemize}
\end{frame}

\begin{frame}{Új műveletek -- adott rangú kulcs meghatározása}
	\begin{itemize}
		\item Cél: $r \leq n$ ranggal rendelkező elem megtalálása a fában
		\item Naiv elgondolás: elkezdem bejárni a fát $<$ szerinti sorrendben, és megállok az $r$-ediknek érintett csúcsnál
		\item<2> $O(h)$ helyett $\Theta(r)$ idejű algoritmus
		\begin{itemize}
			\item Kiegyensúlyozott fa és kellően nagy $r$ estében pedig $r \gg h$
		\end{itemize}
	\end{itemize}
\end{frame}
		
\begin{frame}{Új műveletek -- adott kulcs rangjának meghatározása}
	\begin{itemize}
		\item Naiv elgondolás: elkezdem bejárni a fát $<$ szerinti sorrendben, és megállok, ha $x$ kulcsot érintem
		\item A válasz a bejárás során érintett kulcsok száma lesz
		\item $O(h)$ helyett $O(n)$ idejű algoritmus
	\end{itemize}
	\begin{alertblock}<2>{A megoldás}
Minden csúcs tároljon el kiegészítő információt magáról!
	\end{alertblock}
\end{frame}

\begin{frame}[fragile]{Rendezett-minta fa implementációja}
	\begin{columns}
		\begin{column}{.7\textwidth}
\begin{lstlisting}[
  linebackgroundcolor={%
    \btLstHL<1>{3}
  }]
class Node {
    Object kulcs;
    int kiegeszito;    
    Node* apa;
    Node* bal;
    Node* jobb;
}
\end{lstlisting}
			\vskip 1em
			\begin{block}<2>{Megjegyzés}
				Az extra adattag az új műveletek hatékony (azaz $O(h)$ idejű) implementációját segítik
			\end{block}
		\end{column}
		
		\begin{column}{.4\textwidth}
		\begin{figure}
		\centering
\begin{tikzpicture}[node distance=0cm,outer sep = 0pt]
    \node (A) [box,minimum width=6em] {kulcs};
    \node (H) [box,anchor=north,minimum width=6em,fill=orange!30] at (A.south) {kiegeszito};
    \node (D) [box,anchor=south,minimum width=6em] at (A.north) {apa};
    \node (B) [box,anchor=north west,minimum width=3em] at (H.south west) {bal};
    \node (C) [box,anchor=north east,minimum width=3em] at (H.south east) {jobb};
    \coordinate (E) at (0,2);
    \coordinate (E) at (0,2);
    \node[below left = 1cm of B] (F) {};
    \node[below right = 1cm of C] (G) {};
%    \path (D) edge [->] node[pos=0.5,anchor=-135,inner sep=1pt] {N} (E);
    \path (D) edge [->] (E);
    \path (B) edge [->] (F);
    \path (C) edge [->] (G);
\end{tikzpicture}
		\end{figure}
		\end{column}
	\end{columns}
\end{frame}

\begin{frame}[fragile]{Fabejárások}
	\begin{itemize}
		\item Inorder/preorder/posztorder bejárások
		\item Legegyszerűbb megvalósításuk rekurzióval történik
		\begin{itemize}
			\item Jó azonban tudni, hogy  \href{https://sites.google.com/site/debforit/efficient-binary-tree-traversal-with-two-pointers}{rekurzió nélkül is megtehető mindez}
		\end{itemize}
	\end{itemize}
	\begin{columns}
		\begin{column}{.3\linewidth}
\begin{alltt}
\small
void \textcolor{green}{in}order(x)\{
  if(x==nil)\{return\}
  inorder(x.bal)
  \textcolor{green}{print(x.kulcs)}
  inorder(x.jobb)
\}
\end{alltt}
		\end{column}
		\begin{column}<2->{.3\linewidth}
\begin{alltt}
\small
void \textcolor{green}{pre}order(x)\{
  if(x==nil)\{return\}
  \textcolor{green}{print(x.kulcs)}
  preorder(x.bal)
  preorder(x.jobb)
\}
\end{alltt}
		\end{column}
		\begin{column}<3>{.3\linewidth}
\begin{alltt}
\small
void \textcolor{green}{post}order(x)\{
  if(x==nil)\{return\}
  postorder(x.bal)
  postorder(x.jobb)
  \textcolor{green}{print(x.kulcs)}
\}
\end{alltt}
		\end{column}
	\end{columns}
\end{frame}

\begin{frame}{Teljes bináris fa}
	\begin{itemize}
		\item Olyan bináris fa, amelynek minden belső csúcsának 2 fia van
	\end{itemize}
	\begin{block}{Fában lévő kulcsok száma}
		A fa $i$-edik szintjén $2^i$ csúcs található \\ \pause $\Rightarrow$ a fában $n=\sum\limits_{i=0}^{h} 2^i=2^{h+1}$ csúcsot található\footnote{bizonyítás teljes indukcióval} \\
		\textbf{Állítás}: $h$ magas fában $O(2^h)$ csúcs található \\ \pause
		\textbf{Megfordítva}: $n$ csúcsból álló fa magassága $\Omega(\log n)$
	\end{block}
\end{frame}

\begin{frame}{Bináris keresőfa}
	\begin{block}{Keresőfa tulajdonság}
		A fa minden $x$ csúcsára teljesül, hogy
		\begin{itemize}
		\item $x.bal.kulcs < x.kulcs$ (amennyiben $x.bal != nil$)
		\item $x.kulcs < x.jobb.kulcs $ (amennyiben $x.jobb != nil$)
		\end{itemize}
	\end{block}
\onslide<2>{
	$<$ rendezés tranzitivitásából adódóan
	\begin{figure}
		\centering
		    \begin{tikzpicture}[level/.style={sibling distance = 2cm, level distance = 1cm}]
		    \node [treenode] {53}
		    child[edge from parent path ={(\tikzparentnode) -- (\tikzchildnode.north)}] {
		            node [subtree] {$<53$}
		        }
		    child[edge from parent path ={(\tikzparentnode) -- (\tikzchildnode.north)}] {
		            node [subtree] {$>53$}
		        }
		    ;
		    \end{tikzpicture}
	\end{figure}
}
\end{frame}

\begin{frame}[fragile]{Kulcs keresése fában}
\begin{columns}
\begin{column}{.5\linewidth}
\begin{alltt}
{\scshape FábanKeres}(x, k) \{
   if x == nil or k == x.kulcs
      return x

   if k < x.kulcs
      {\scshape FábanKeres}(x.bal, k)
   else 
      {\scshape FábanKeres}(x.jobb, k)
 \}
\end{alltt}
\end{column}
\begin{column}<2>{.4\linewidth}
$h$ magas fa esetén $O(h)$
\end{column}
\end{columns}
\end{frame}

\begin{frame}{Elem beszúrása bináris keresőfába}
  \begin{itemize}
    \item A keresőfa tulajdonság fenntartása mellett levélként szúrunk be
	\item $h$ magas fa esetén $O(h)$ idejű
  \end{itemize}
  \begin{example}
 	\begin{figure}
 	    \begin{tikzpicture}
 		    \node[treenode]{20};
 		\end{tikzpicture}
 	\end{figure}

 	{\scshape Beszúr(7)}

	\begin{figure}
		\begin{tikzpicture}
			\node[treenode] {20}
    		child{ node[treenode] {7}}
			child[missing];
 		\end{tikzpicture}
 	\end{figure}
  \end{example}
\end{frame}

\begin{frame}{Elem törlése bináris keresőfából}
\pause
	\begin{itemize}
		\item 3 esetet különböztetünk meg $x$ csúcs törlése kapcsán
		\begin{enumerate}
			\item $x$-nek nincs gyereke
			\begin{itemize}
				\item $x$ apjának az $x$-re vonatkozó mutatóját $nil$-re állítjuk
			\end{itemize}
			\item $x$-nek pontosan egy gyereke van
			\begin{itemize}
				\item $x$ apját ''átkötjük'' $x$ egyedüli fiához
			\end{itemize}
			\item $x$-nek 2 gyereke van
			\begin{itemize}
				\item $x$-et megelőzőjével (bal oldali részfájának maximális elemével) helyettesítjük
			\end{itemize}
		\end{enumerate}
		\item $h$ magas fa esetén $O(h)$ idejű
	\end{itemize}
\end{frame}

\end{document}