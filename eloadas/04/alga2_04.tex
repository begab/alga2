\documentclass{beamer}
%\documentclass[handout]{beamer}
\usepackage[hungarian]{babel}
\uselanguage{hungarian}
\languagepath{hungarian}
\deftranslation[to=hungarian]{Theorem}{T\'etel}
\deftranslation[to=hungarian]{Example}{P\'elda}
\deftranslation[to=hungarian]{Definition}{Defin\'ici\'o}
%\usepackage[magyar]{babel}
\usepackage[utf8]{inputenc}
\usepackage[T1]{fontenc}
\usepackage{beamerthemesplit}
\usepackage{pgf,pgffor,pgfplots}
\pgfplotsset{compat=1.15}
\usepackage{subfig}
\usepackage{xcolor}
\usepackage{listings}
\usepackage{lstlinebgrd}
\AtBeginEnvironment{figure}{\setcounter{subfigure}{0}}
\makeatletter
%%%%%%%%%%%%%%%%%%%%%%%%%%%%%%%%%%%%%%%%%%%%%%%%%%%%%%%%%%%%%%%%%%%%%%%%%%%%%%
%
% \btIfInRange{number}{range list}{TRUE}{FALSE}
%
% Test in int number <number> is element of a (comma separated) list of ranges
% (such as: {1,3-5,7,10-12,14}) and processes <TRUE> or <FALSE> respectively

\newcount\bt@rangea
\newcount\bt@rangeb

\newcommand\btIfInRange[2]{%
    \global\let\bt@inrange\@secondoftwo%
    \edef\bt@rangelist{#2}%
    \foreach \range in \bt@rangelist {%
        \afterassignment\bt@getrangeb%
        \bt@rangea=0\range\relax%
        \pgfmathtruncatemacro\result{ ( #1 >= \bt@rangea) && (#1 <= \bt@rangeb) }%
        \ifnum\result=1\relax%
            \breakforeach%
            \global\let\bt@inrange\@firstoftwo%
        \fi%
    }%
    \bt@inrange%
}
\newcommand\bt@getrangeb{%
    \@ifnextchar\relax%
        {\bt@rangeb=\bt@rangea}%
        {\@getrangeb}%
}
\def\@getrangeb-#1\relax{%
    \ifx\relax#1\relax%
        \bt@rangeb=100000%   \maxdimen is too large for pgfmath
    \else%
        \bt@rangeb=#1\relax%
    \fi%
}

%%%%%%%%%%%%%%%%%%%%%%%%%%%%%%%%%%%%%%%%%%%%%%%%%%%%%%%%%%%%%%%%%%%%%%%%%%%%%%
%
% \btLstHL<overlay spec>{range list}
%
% TODO BUG: \btLstHL commands can not yet be accumulated if more than one overlay spec match.
% 
\newcommand<>{\btLstHL}[1]{%
  \only#2{\btIfInRange{\value{lstnumber}}{#1}{\color{orange!30}\def\lst@linebgrdcmd{\color@block}}{\def\lst@linebgrdcmd####1####2####3{}}}%
}%
\makeatother

\usepackage{hyperref}
\hypersetup{
    colorlinks = true,
    linkcolor = blue,
    urlcolor  = blue,
    citecolor = blue,
    linkbordercolor = {white},
}
\usepackage{alltt}
\usepackage{tikz}
\usetikzlibrary{trees}
\usetikzlibrary{shapes,shapes.geometric,shapes.multipart}
\usetikzlibrary{calc,chains,arrows,positioning}
\tikzset{
  box/.style={draw, fill=pink!10, minimum width=5em, text centered, minimum height=2.5em},
  treenode/.style = {align=center, inner sep=0pt, text centered, font=\sffamily},
  arn_n/.style = {treenode, circle, white, font=\sffamily\bfseries, draw=black, fill=black, text width=1.5em},
  arn_r/.style = {treenode, circle, red, draw=red, text width=1.5em, very thick}
}
\usetheme{Warsaw}
\institute{Szegedi Tudományegyetem}
\pgfdeclareimage[height=0.55cm]{institution-logo}{../szte_logo}
\logo{\pgfuseimage{institution-logo}}

\title{Algoritmusok és adatszerkezetek II.}
\subtitle{Piros-fekete fák}
\date{}

\begin{document}

\maketitle

\begin{frame}{Piros-fekete fák tulajdonságai}
	\begin{enumerate}
		\item Minden csúcs színe piros vagy fekete
		\item A gyökér színe fekete
		\item Minden levele\footnote{levelek alatt itt most az ''őrszemeket'' értjük} fekete
		\item A piros csúcsoknak \textbf{kizárólag} fekete színű gyerekeik vannak
		\item Bármely csúcsból azonos számú fekete csúcs érintésével jutunk el bármelyik levélbe
	\end{enumerate}

	\pause
	\begin{theorem}
		Bármely $n$ kulcsú piros-fekete fa magassága legfeljebb $2\log(n+1)$.
	\end{theorem}
\end{frame}

\begin{frame}{Fekete-magasság}
	\begin{itemize}
		\item $fm(x)$ jelölje az $x$ csúcsból induló, bármely levélig vezető úton található,
		($x$-en kívüli) fekete csúcsok számát
	\end{itemize}
\end{frame}

\begin{frame}[fragile]{Piros-fekete fa implementációja}
	\begin{columns}
		\begin{column}{.7\textwidth}
\begin{lstlisting}[
  linebackgroundcolor={
    \btLstHL<1->{3}
  }]
class Node {
    Object kulcs;
    boolean fekete;
    Node *apa;
    Node *bal;
    Node *jobb;
}
\end{lstlisting}
			\begin{block}<2>{Megjegyzés}
				Az eddigi kiegészítő információk közül a legolcsóbb (csupán 1 bit)
			\end{block}

		\end{column}
		
		\begin{column}{.4\textwidth}
		\begin{figure}
		\centering
\begin{tikzpicture}[node distance=0cm,outer sep = 0pt]
    \node (A) [box,minimum width=6em] {kulcs};
    \node (H) [box,anchor=north,minimum width=6em,fill=orange!30] at (A.south) {fekete};
    \node (D) [box,anchor=south,minimum width=6em] at (A.north) {apa};
    \node (B) [box,anchor=north west,minimum width=3em] at (H.south west) {bal};
    \node (C) [box,anchor=north east,minimum width=3em] at (H.south east) {jobb};
    \coordinate (E) at (0,2);
    \coordinate (E) at (0,2);
    \node[below left = 1cm of B] (F) {};
    \node[below right = 1cm of C] (G) {};
%    \path (D) edge [->] node[pos=0.5,anchor=-135,inner sep=1pt] {N} (E);
    \path (D) edge [->] (E);
    \path (B) edge [->] (F);
    \path (C) edge [->] (G);
\end{tikzpicture}
		\end{figure}
		\end{column}
	\end{columns}
\end{frame}

\begin{frame}{Összegzés}
	\begin{itemize}
		\item A bináris keresőfák műveletei $O(h)$ idejűek
		\item Legrosszabb esetben azonban $n$ is lehet a fák magassága ($\Theta(\log n)$ helyett)
		\item Kiegyensúlyozott keresőfák használatával garantálható, hogy a keresőfa kiegyensúlyozottsága sose romoljon el ''túlságosan''
	\end{itemize}
\end{frame}

\end{document}