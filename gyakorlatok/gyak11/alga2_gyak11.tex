\documentclass[12pt]{article}
\usepackage[hungarian]{babel}
\selectlanguage{hungarian}
\usepackage[utf8]{inputenc}
\usepackage[T1]{fontenc}
\usepackage{blkarray}

\title{11. gyakorlat -- Korlátozás és szétválasztás módszere}
\date{}
\begin{document}

\maketitle

\noindent 1. Vegyük az következő hozzárendelési feladatot\footnote{A B\&B módszeren túl Magyar módszerrel (aka.~Kuhn–Munkres algoritmus) is megoldható a probléma $O(n^3)$ időben.}. Adott $n$ munkás és $n$ elvégzendő feladat. Minden munkás más-más költségen végez el egy feladatot. Rendeljük hozzá az elvégzendő feladatokat úgy a munkásokhoz, hogy azokat a legkisebb összköltséggel végezzék el. Például a

\[
C = \begin{blockarray}{ccccc}
a & b & c & d \\
\begin{block}{(cccc)l}
  6 & \underline{2} & 8 & 7 & I. \\
  2 & 1 & 5 & \underline{4} & II.\\
  \underline{1} & 3 & 2 & 5 & III. \\
  4 & 2 & \underline{1} & 3 & IV. \\
\end{block}
\end{blockarray}
 \]

\noindent költségmátrix esetében az $a,b,c,d$ feladatokat rendre a III., I., IV., valamint II. munkás végezze el $1+2+1+4=8$ összköltséggel, melyről belátható, hogy egy minimális hozzárendelést eredményez.

Fontos kikötés, hogy minden munkásnak \textbf{pontosan} egy feladatot kell elvégezzen, tehát pl.~a $c$ és a $d$ feladatok nem kerülhetnek egyidejűleg kiosztásra a IV. munkás számára.

A B\&B használata során a tényleges $f$ (össz)költségfüggvényt optimista módon 
(alulról) becslő $g$ függvényre van szükség. $g\le f^*$ értékét egy 
részhozzárendeléshez határozzuk meg mohó módon, vagyis a még ki nem osztott 
munkákra vonatkozóan átmenetileg tegyük fel, hogy nem kell teljesüljön az egy 
munkás-egy feladat megkötés, vagyis a ki nem osztott munkák közül egy munkásra 
több feladat is kiosztható.

Tehát pl.~$g$ kezdeti értéke $g=1+1+1+3=6$ (minden oszlop minimumának összegét 
véve), azaz akárhogy is rendeljük hozzá a feladatokat a munkásokhoz, az 
összköltség legalább $6$ lesz.

A B\&B a feladatot leíró állapotteret a $g$ függvény figyelembe vétele mellett 
járja be/szűri meg.
\end{document}